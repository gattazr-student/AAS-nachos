\documentclass[a4paper,10pt]{article}

% Encodage and langue
\usepackage[utf8]{inputenc}
\usepackage[T1]{fontenc}
\usepackage[francais]{babel}

% Redéfinition des marges
\usepackage[top=2cm, bottom=2cm, left=2.0cm, right=2cm]{geometry}

% Package pour insertion de code
\usepackage{listings} % Insertion de code
\usepackage{graphicx} % insertion d'images
% \usepackage{float} % Placement d'éléments
\usepackage[dvipsnames]{xcolor} % colors
\usepackage{url} % Insertion d'url url

% \usepackage{amsfonts} % pour utiliser les symboles de ensembles (reel...autre)
% \usepackage{amsmath} %debut des package pour utiliser les formules de math
% \usepackage{amssymb}
% \usepackage{mathrsfs}
%\usepackage{wrapfig}


% Listing config
\lstset{
    language=C++,
    basicstyle=\normalsize, % ou ça==> basicstyle=\scriptsize,
    % upquote=true,
    aboveskip={1.5\baselineskip},
    columns=fullflexible,
    showstringspaces=false,
    extendedchars=true,
    breaklines=true,
    showtabs=false,
    showspaces=false,
    showstringspaces=false,
    identifierstyle=\ttfamily,
    keywordstyle=\color[rgb]{0,0,1},
    commentstyle=\color[rgb]{0.133,0.545,0.133},
    stringstyle=\color[rgb]{0.627,0.126,0.941},
}

\title{Step 2 : Exceptions, Entrées et sorties utilisateurs}
\author{Tanguy MATHIEU, Florian POPEK, Rémi GATTAZ, Jordan ELLAPIN}


\begin{document}
    \maketitle
    \tableofcontents
    \newpage

    \section{Objectifs}
    Au cours de cette étape, nous avons eu les objectifs suivant :
    \begin{itemize}
        \item Comprendre le fonctionnement des appels systèmes
        \item Comprendre le fonctionnement de la console asynchrone
        \item Créer une console synchrone
        \item Ajouter des appels systèmes pour gérer les entrées sorties
    \end{itemize}

    Chacun de ces objectifs va être détaillé dans une partie.


    \section{Appels systèmes}

    \section{Console asynchrone}

    \section{Console synchrone}

    \section{Entrées/Sorties}

    \subsection{GetChar/PutChar}
    \subsection{GetString/PutString}
    \subsection{GetInt/PutInt}


    \subsection{FEOF}

\end{document}
